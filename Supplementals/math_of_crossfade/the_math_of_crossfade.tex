\documentclass{article}[12pt]
\usepackage[letterpaper, margin=1in]{geometry}
\usepackage{amsmath}
\usepackage{amsfonts}

\setlength{\parskip}{1em}
\linespread{1.3}
\newcommand{\supt}[1]{^{\text{#1}}}
\newcommand{\gen}[2]{\Gamma_{(#1,#2)}}
\newcommand{\group}[1]{M_{#1 \times #1}(\mathbb{Z}/2\mathbb{Z})}

\begin{document}

\section*{Introduction}

The goal of this document is to provide a mathematical model for the game CrossFade.  The game is played on a square grid of tiles, which we will call the board.  In principle, the board can be any size.  In examples, we will use small integers, but when doing work in the abstract, we will discuss an arbitrary board of size $n > 1$, which holds $n^2$ tiles.  Each tile has a light which can be on or off.  A configuration of the board is a collection of on/off values corresponding to each tile.

The player moves by selecting tiles (also called ``moving on" a tile).  Selecting a tile flips the on/off value of every tile in the selected tile's row and column (including the selected tile).  The player's goal is to turn off all the lights.  Specifically, a game of CrossFade consists of a starting board configuration and a sequence of moves that results in the all-off board configuration, which we will call the blank board.  We refer to starting configurations where a win is possible as winnable configurations, or solvable configurations.

Ideally, our model will provide insights as to how to design effective CrossFade levels.  With this in mind, we state the following questions in advance.

\begin{itemize}

\item Are all initial configurations solvable?  If not, what characterizes solvable configurations?

\item Does order matter when selecting tiles?

\item Does the size of the board change the character of the set of solvable states?

\end{itemize}

\section{The Matrix Model}

To attain a more concrete mathematical model for CrossFade, we will represent board configurations with matrices.  Specifically, we will use matrices whose entries are 1's and 0's: 1 for lights that are on, 2 for lights that are off.

Matrices of this type have a natural entry-by-entry addition, but with a twist: if two 1\'s are added, the resulting entry is a 0, not a 1.  In truth, for a given board side-length $n$, configurations of the board are elements of the additive matrix group $\group{n}$; that is, $n$ by $n$ matrices over the field $\mathbb{Z}/2\mathbb{Z}$.  This is fancy algebra-speak for ``$n$ by $n$ integer matrices mod 2."  The significance of being able to add two configurations will be explored later.  For now, we will use matrix addition to model making moves.

One thing to note about matrices in $\group{n}$ before we begin modelling moves: they are each their own additive inverse.  That is, add any matrix to itself, and you will get a matrix that has 0's wherever the original had 0's and 2's wherever the original had 1's.  Since we our integer arithmetic is mod 2, all the 2's are actually 0's.  In other words, the result of adding a matrix to itself is the identity matrix, $I$, which corresponds to the blank board state.  This allows us to simplify sums of matrices a fair bit.  Any term that appears twice in a sum cancels itself out, so any matrix sum can be reduced to a sum of unique terms.  This will be useful later!

\section{Modelling Moves and Games}

Let's talk about modelling moves.  Moving on the tile in row $i$, column $j$ flips the value of all the tiles in the $i\supt{th}$ row and $j\supt{th}$ column.  In terms of matrix addition, this means adding a 1 to the entries in that row and column.  Moving on the $(i,j)\supt{th}$ tile, then, is equivalent to adding a matrix that is 0's everywhere except the $i\supt{th}$ row and $j\supt{th}$ column, which are all 1's.  We will refer to such matricies as $\gen{i}{j}$, or simply $\Gamma$ when speaking generally.  Recall that every matrix in $\group{n}$ is its own additive inverse; this encapsulates the fact that any move in CrossFade may be undone by moving on the same tile again.

Given an initial board configuration represented by the matrix $M_0$, a winning game of CrossFade would consist of a series of moves, each represented by $\gen{i}{j}$ for some $i, j < n$.  We can represent this series of moves by a sum that yields the identiy matrix:

\begin{equation}
\label{model_game_1}
M_0 + \Gamma_1 + \Gamma_2 + \cdots + \Gamma_m = I
\end{equation}

Immediately, we can draw several conclusions from this model.  First, note that matrix addition is \emph{commutative}, meaning we can freely rearrange the terms, just as we can with a finite sum of integers.  We have an answer to one of our questions: it does \emph{not} matter the order in which moves are made.  A collection of moves may be made in any order and it will produce the same result.  Also recall the special property of matrix addition in $\group{n}$: any sum may be reduced to a sum of unique terms.  This gives us another insight about CrossFade games: in the shortest potential winning game, any given tile is moved on at most once.

Since any sequence of moves is reversible by repeating each move (it doesn't even matter in what order), we can equally easily think of winnable configurations as those from which we can reach the identiy and those we can reach from the identity by summing together some combination of $\Gamma$ matrices.  This is often easier.  Equation \ref{model_game_1} can be restate as:

\begin{equation}
\label{model_game_2}
\Gamma_1 + \Gamma_2 + \cdots + \Gamma_m = M
\end{equation}

That is, every winnable configuration matrix $M$ can be expressed as a sum of unique matrices $\Gamma$.  This is a good start on characterizing winnable states.  We may borrow more from algebra, and assert that the set of winnable states is the \emph{subgroup generated by $\{\Gamma\}$}, written $\langle\Gamma\rangle$.  This refers to the subset of $\group{n}$ consisting of matrices which can be expressed as a sum of $\Gamma$ matrices.  This set has group structure, meaning it's closed under addition.  In other words, \emph{the sum of winnable states is winnable}.

One might reasonably ask what it even \emph{means} to add two configurations.  It makes sense as far as matrix arithmetic goes, and representing moves as matrix addition seems reasonable, but what is the game-relevance of summing two configurations?  Since any winnable matrices can be expressed as a sum of $\Gamma$ matrices, we can restate any sum of winnable configurations as a longer sum of move matrices.  This means that matrices in $\langle\Gamma\rangle$ represent not only winnable states, but also a handy shorthand for a sequence of moves.  It will be very helpful to freely add together various winnable matrices, knowing that the sum itself is still a sum of $\Gamma$ matrices.

\end{document}