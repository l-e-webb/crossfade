\documentclass{article}[12pt]
\usepackage[letterpaper, margin=1in]{geometry}
\usepackage{amsmath}
\usepackage{amsfonts}

\setlength{\parskip}{1em}
\linespread{1.3}
\newcommand{\supt}[1]{^{\text{#1}}}
\newcommand{\gen}[2]{\Gamma_{(#1,#2)}}
\newcommand{\basis}[2]{\beta_{(#1,#2)}}
\newcommand{\group}[1]{M_{#1 \times #1}(\mathbb{Z}/2\mathbb{Z})}
\newcommand{\subgroup}[1]{\langle\{\Gamma\}_{#1}\rangle}
\newcommand{\refx}[1]{(\ref{#1})}
\renewcommand*{\thefootnote}{\fnsymbol{footnote}}

\begin{document}

\section*{Introduction}

The goal of this document is to provide a mathematical model for the game CrossFade.  The game is played on a square grid of tiles, which we will call the board.  In principle, the board can be any size.  In examples, we will use small integers, but when doing work in the abstract, we will discuss an arbitrary board of size $n > 1$, which holds $n^2$ tiles.  Each tile has a light which can be on or off.  A configuration of the board is a collection of on/off values corresponding to each tile.

The player moves by selecting tiles (also called ``moving on" a tile).  Selecting a tile flips the on/off value of every tile in the selected tile's row and column (including the selected tile).  The player's goal is to turn off all the lights.  Specifically, a game of CrossFade consists of a starting board configuration and a sequence of moves that results in the all-off board configuration, which we will call the blank board.  We refer to starting configurations where a win is possible as winnable configurations, or solvable configurations.

Ideally, our model will provide insights as to how to design effective CrossFade levels.  With this in mind, we state the following questions in advance.

\begin{itemize}

\item Are all initial configurations solvable?  If not, what characterizes solvable configurations?

\item Does order matter when selecting tiles?

\item Does the size of the board change the character of the set of solvable states?

\end{itemize}

\section{The matrix model}

To attain a more concrete mathematical model for CrossFade, we will represent board configurations with matrices.  Specifically, we will use matrices whose entries are 1's and 0's: 1 for lights that are on, 2 for lights that are off.

Matrices of this type have a natural entry-by-entry addition, but with a twist: if two 1\'s are added, the resulting entry is a 0, not a 1.  In truth, for a given board side-length $n$, configurations of the board are elements of the additive matrix group $\group{n}$; that is, $n$ by $n$ matrices over the field $\mathbb{Z}/2\mathbb{Z}$.  This is fancy algebra-speak for ``$n$ by $n$ integer matrices mod 2."  The significance of being able to add two configurations will be explored later.  For now, we will use matrix addition to model making moves.

One thing to note about matrices in $\group{n}$ before we begin modelling moves: they are each their own additive inverse.  That is, add any matrix to itself, and you will get a matrix that has 0's wherever the original had 0's and 2's wherever the original had 1's.  Since we our integer arithmetic is mod 2, all the 2's are actually 0's.  In other words, the result of adding a matrix to itself is the identity matrix, $I$, which corresponds to the blank board state.  This allows us to simplify sums of matrices a fair bit.  Any term that appears twice in a sum cancels itself out, so any matrix sum can be reduced to a sum of unique terms.  This will be useful later!

\section{Modelling moves and games}

Let's talk about modelling moves.  Moving on the tile in row $i$, column $j$ flips the value of all the tiles in the $i\supt{th}$ row and $j\supt{th}$ column.  In terms of matrix addition, this means adding a 1 to the entries in that row and column.  Moving on the $(i,j)\supt{th}$ tile, then, is equivalent to adding a matrix that is 0's everywhere except the $i\supt{th}$ row and $j\supt{th}$ column, which are all 1's.  We will refer to such ``move matricies" as $\gen{i}{j}$, or simply $\Gamma$ when speaking generally.  The set of all move matrices for an $n$ by $n$ board will be referred to as $\{\Gamma\}_n$.

A note on notation: for most of this document, we are, tacitly or explicitly, discussing an $n$ by $n$ board.  When we use notation such as $\gen{i}{j}$, $\Gamma$, we are talking about $n$ by $n$ matrices for an arbitrary but fixed $n$.  We never need to compare move or configuration matrices that are of different sizes.

Now that we have a model for moves, we can apply it to entire CrossFade games.  Given an initial board configuration represented by some matrix $M_0$, a winning game of CrossFade would consist of a series of moves, each represented by $\gen{i}{j}$ for some $i, j < n$.  We can represent this series of moves by a sum that yields the identiy matrix:

\begin{equation}
\label{model_game_1}
M_0 + \Gamma_1 + \Gamma_2 + \cdots + \Gamma_m = I
\end{equation}

Immediately, we can draw several conclusions from this model.  First, note that matrix addition is \emph{commutative}, meaning we can freely rearrange the terms, just as we can with a finite sum of integers.  We have an answer to one of our questions: it does \emph{not} matter the order in which moves are made.  A collection of moves may be made in any order and it will produce the same result.  Also recall the special property of matrix addition in $\group{n}$: any sum may be reduced to a sum of unique terms.  This gives us another insight about CrossFade games: in the shortest potential winning game, any given tile is moved on at most once.

This is another useful fact about matrix addition in $\group{n}$ that we noted earlier: every matrix is its own additive inverse.  This encapsulates the fact that any move in CrossFade may be undone by moving on the same tile again.Since any sequence of moves is reversible by repeating each move (it doesn't even matter in what order), we can equally easily think of winnable configurations as those from which we can reach the identiy matrix and those we can reach from the identity matrix by summing together some combination of $\Gamma$ matrices.  This is often easier.  Equation \refx{model_game_1} can be restate as:

\begin{equation}
\label{model_game_2}
\Gamma_1 + \Gamma_2 + \cdots + \Gamma_m = M
\end{equation}

That is, every winnable configuration matrix $M$ can be expressed as a sum of unique matrices $\Gamma$.  We will call any representation of a matrix as a sum of $\Gamma$ matrices as a \emph{$\Gamma$-representation} of that matrix.  If the terms in the sum are unique, we will call that $\Gamma$-representation \emph{irreducible}.  Note that any $\Gamma$-representation may be reduced to an irreducible one.  To borrow more from algebra, and assert that the set of winnable states is the \emph{subgroup generated by $\{\Gamma\}_n$}, written $\subgroup{n}$.  This refers to the subset of $\group{n}$ consisting of matrices which have $\Gamma$-representations\footnote{In fact, this is the motivation behind the name $\Gamma$.  Group generators are traditionally represented by the lower-case gamma, $\gamma$.  Since matrices typically are given upper-case names, we a priori named our generators using an upper-case gamma, $\Gamma$}.  This set has group structure, meaning it's closed under addition.  In other words, \emph{the sum of winnable configurations is winnable}.

One might reasonably ask what it even \emph{means} to add two configurations.  It makes sense as far as matrix arithmetic goes, and representing moves as matrix addition seems reasonable, but what is the game-relevance of summing two configurations?  Since any winnable matrix has a $\Gamma$-representation, we can restate any sum of winnable configurations as a longer sum of move matrices.  This greatly increases the utility of the model represented by the group $\subgroup{n}$.  Any element $M$ in $\subgroup{n}$, along with its $\Gamma$-representation(s), can be seen as a) a winnable configuration, b) a sequence of moves that goes from that winnable configuration to the blank board (ie a winning game), c) a sequence of moves that goes from the blank board to that winnable configuration, \emph{or} d) a sequence of moves that goes from some other configuration $M_0$ to a new configuration, $M_0 + M$.  In other words, the $\Gamma$-representation of $M$ is a sequence of moves that applies a transformation to the board equivalent to adding the matrix $M$.  This is very useful!  It allows us to think beyond a single move, and use matrices in $\subgroup{n}$ to represent a known sequences of moves with a specific effect.  We will explore this more later.

\section{Studying the group $\subgroup{n}$}

We now have a sound model for mathematically representing CrossFade board configurations, moves, and games.  Already, this has provided some insights as to how the game functions.  The biggest questions are yet to be answered: are all configurations winnable?  If not, which ones are?

To answer these questions, we turn our attention to the subgroup $\subgroup{n}$.  We know that inside the whole space of potential configurations of an $n$ by $n$ board, $\group{n}$, $\subgroup{n}$ is a subgroup that consists of all the winnable states.  One question we might ask is how large is $\subgroup{n}$.  This is difficult to determine, but we know that every element of $\subgroup{n}$ by definition has a $\Gamma$-representation.  Moreover, we know that any such representation may be reduced to an irreducible $\Gamma$-representation.  So how many possible irreducible $\Gamma$-representations are there?  We turn to the world of combinatorics.

To start, note that there are $n^2$ $\Gamma$ matrices: a $\gen{i}{j}$ for each tile in an $n$ by $n$ board.  How many different ways can you combine $n^2$ items?  The notation $n \choose k$ represents the number of $k$-sized collections you can make out of an $n$ element set.  Applying this to our problem, $n^2 \choose k$ would represent the number of $k$-term sums we could make with unique $\Gamma$ matrices.  To count all the possible sums, we want to calculate $n^2 \choose k$ for each $k$ from 0 to $n^2$.

\begin{equation}
\label{subgroup_size}
\text{\# of irreducible $\Gamma$-representations} = \sum_{k = 0}^{n^2} {n^2 \choose k}
\end{equation}

To evaluate the sum on the right, we will use the \emph{binomial theorem}, which states that for two real variables $x$ and $y$:

\begin{equation}
\label{binomial_theorem}
(x + y)^n = \sum_{k = 0}^n {n \choose k}x^ky^{n - k}
\end{equation}

The sum on the right in \refx{binomial_theorem} looks suspiciously like the sum on the right in \refx{subgroup_size}, but we cannot immediately use the former to evaluate the latter.  To do that, we first consider a special case of the binomial theorem where $x$ and $y$ are both 1.  Substituting 1 in for both variables in \refx{binomial_theorem} yields this equality:

\begin{equation}
\label{special_binomial}
2^n = \sum_{k = 0}^n {n \choose k}
\end{equation}

Now we are almost there.  The only thing that remains is to make the somewhat confusing substitution of $n^2$ for $n$.  (In the binomial theorem, the $n$ is just an arbitrary integer, whereas in our environment, $n$ represents board side-length.)  Combining \refx{subgroup_size} and \refx{special_binomial} with this substitution yields:

\begin{equation}
\begin{aligned}
\text{\# of irreducible $\Gamma$-representations} &= \sum_{k = 0}^{n^2} {n^2 \choose k}\\
&= 2^{n^2}
\end{aligned}
\end{equation}

There are $2^{n^2}$ possible sums that consist of elements of $\{\Gamma\}_n$ used at most once.  First, let's note that this number is incredibly large.  For $n = 5$, it is over 33 million.  However large, this number by itself doesn't have much significance at a glance.  Let's compare it to the size of the entire group $\group{n}$.  What is the total number of configuration states of an $n$ by $n$ board?  Well, there are $n^2$ tiles, and each can be on or off.  The number of ways you can make $n$ choices with $k$ options each is $k^n$.  So the number of ways you can make $n^2$ choices with 2 options each is... $2^{n^2}$.  The same value as the number of possible irreducible $\Gamma$-representations.  Coincidence?  Not likely!  Does this mean that the size of the subgroup $\subgroup{n}$ is the same as the size of the whole group, which would imply that it \emph{is} the whole group, meaning all configurations are winnable?

Unfortunately, the answer is ``not necessarily."  We counted the number of sums you could make out of $\Gamma$ matrices where each term was unique, but that is merely an upper bound on the size of $\subgroup{n}$.  The size of $\subgroup{n}$ is $2^{n^2}$ if and only if \emph{each irreducible $\Gamma$-representation produces a unique configuration.}  In other words, if no element of $\subgroup{n}$ has multiple distict irreducible $\Gamma$-representations.

So, is that true?  Do distinct $\Gamma$-representations necessarily produce distinct results?  The answer is, again, ``not necessarily."

\section{Uniqueness of $\Gamma$-representation in $\subgroup{n}$}

Our goal now is to determine if and when it is true that elements of $\subgroup{n}$ have unique irreducible $\Gamma$-representations.  If the irreducible $\Gamma$-representations are unique, then the upper bound for the size of $\subgroup{n}$ that we calculated in the previous section is in fact the actual size.  In that case, $\subgroup{n}$ contains every possible configuration, meaning every configuration is winnable.  If irreducible $\Gamma$-represenations are \emph{not} unique, then $\subgroup{n}$ does \emph{not} contain every possible configuration.  Thus, the question of what configurations are solvable hinges on the question of unique irreducible $\Gamma$-representations.  It turns out that the answer to this question changes based on not the size of $n$, but whether $n$ is even or odd.

\subsection{For odd $n$}

We now consider only $n$ by $n$ boards where $n$ is odd.  Consider the result of moving on every tile in a row or column.  This is equivalent to the following sum (here of the $k\supt{th}$ row):

\begin{equation}
\label{sum_row}
\gen{k}{1} + \gen{k}{2} + \cdots + \gen{k}{n} = M
\end{equation}

What is the resulting configuration $M$?    First, let's consider an entry in $M$ not on the $k\supt{th}$ row.  For some entry at position $(i,j)$ where $i \neq k$ (or, equivalently, for a some tile not on the $k\supt{th}$ row), there is only one move matrix in that sum that has a non-zero value at that entry: $\gen{k}{j}$.  All other moves are on other columns, and there are no moves on the $i\supt{th}$ row.  Thus, the value in $M$ at position $(i,j)$ is 1.  Now consider a an entry in the $k\supt{th}$ row.  \emph{Every} term in the sum has a value of 1 for this tile, so the resulting value in $M$ at position $(k,j)$ (for any $j$) is $n$ mod 2.  Since $n$ is assumed to be odd, this value is also a 1.  In other words, $M$ is a matrix composed entirely of 1's, representing a board where every tile is set to on.

Note that this is true for an arbitrary row $k$.  It's also true for a column: if you just reverse all the coordinates in the above paragraph, the reasoning still holds.  On an odd-sided board, \emph{moving on each tile in a row or column will turn on every tile on the board.}  Go ahead, try it!  Open up CrossFade to an empty board and move on each tile in a row or column.  The whole board lights up!  Neat, huh?

It is important to note that the sum in \refx{sum_row} is irreducible, and that this is true for any row and column.  This means that the matrix $M$ in \refx{sum_row} (ie the matrix of all 1's) has a number of irreducible $\Gamma$-representations.  At least $2n$: one for each row and column.  So for odd-sided boards, it is \emph{not} true that irreducible $\Gamma$ representations are unique.  Thus, $\subgroup{n}$ is a proper subgroup of $\group{n}$, and not all configurations are solvable.

A final note on the process of moving on every tile in a row or column.  Recall that matrices in $\subgroup{n}$ and their $\Gamma$-representations have multiple interpretations.  We may think of moving on each tile in a row or columnas a sequence that turns on every light on a blank board, but we can also view it more generally as a sequence that will reverse the value at every tile on the board.  Now that we know how to perform this operation, we may apply it freely in play and design.  In the matrix world, instead of only adding $\Gamma$ matrices, we can now add the matrix of all ones, secure in the knowledge that this addition can be decomposed easily into a sum of $\Gamma$ matrices.

\subsection{For even $n$}

We have shown that irreducible $\Gamma$-representations are not unique in $\subgroup{n}$ for odd $n$, but the oddness featured prominently in that demonstration.  What about for even $n$?  Again, we will consider the effect of a specific sequence of moves.  What happens if we move on each tile in a certain row and a certain column (moving on the intersection once, not twice)?  You can go ahead and test it on an even board in CrossFade, but we will consider it in the general case.

Fix a row $k$ and a column $l$, and consider the sum:

\begin{equation}
\label{sum_row_and_col}
\gen{k}{1} + \gen{k}{2} + \cdots + \gen{k}{n} + \gen{1}{l} + \gen{2}{l} + \cdots \gen{k - 1}{l} + \gen{k + 1}{l} + \cdots + \gen{n}{l} = M
\end{equation}

Note that the term $\gen{k}{l}$ is included in the first $n$ terms, but not in the second.  We move on each tile in the $k\supt{th}$ row or $l\supt{th}$ column exactly once.

Again, we will consider the value at various entries of the resulting matrix $M$.  For $(i,j)$ where $i \neq k$ and $j \neq l$, which terms in that sum have a non-zero value in the $(i,j)\supt{th}$ entry?  Two: $\gen{k}{j}$ and $\gen{i}{l}$.  The resulting value, then, is 2 mod 2, which is 0.  Next, consider some entry on row $k$ but not on column $l$.  How many terms in the sum have a non-zero value for this entry?  Every move on row $k$ does, but none of the others.  The value in that entry of $M$, then, is $n$ mod 2.  Since $n$ is now assumed to be even, this is also 0.  This same logic can be applied to any entry in column $l$ but not row $k$: it is affected by terms of the form $\gen{i}{l}$, but not others.

So far, it seems like all the entries in $M$ are zero.  What about the entry at position $(k,l)$?  How many terms affect this entry?  All of them, actually: every term is either in row $k$ or column $l$.  There are $2n - 1$ terms on the left in \refx{sum_row_and_col} (remember, we don't double add $\gen{k}{l}$).  The resulting value in that entry, then is $2n - 1$ mod 2.  Since $2n - 1$ is odd, this value is 1.

This is very significant.  Since we did this in generality, we can do this with any row/column pair to produce a sum of unique $\Gamma$ that results in a matrix that is all 0's except one position with a 1.  Call such matrices $\basis{i}{j}$\footnote{$\beta$ for \emph{basis}.}, where $\basis{i}{j}$ is all zeros except position $(i,j)$, which is 1.  These matrices generate the entire group $\group{n}$.  Every configuration matrix can be expressed as a unique sum of unique matrices $\basis{i}{j}$: for each 1 in a terget configuration's matrix, add the basis matrix with a 1 in the corresponding position.  Every matrix is a sum of basis matrices, and basis matrices have $\Gamma$-representations.  Therefore: every matrix has a $\Gamma$-representation.  For even $n$, all configurations are solvable, $\subgroup{n}$ does indeed have size $2^{n^2}$, and is equal to the entire group $\group{n}$.

In game-turns, this means total omnipotence.  Remembering that $\basis{i}{j}$ can be expressed by moving on each tile in row $i$ and column $j$ (being sure not to double-move on tile $(i,j)$), we can view this sequence of moves as a transformation which reverses the value of one tile, but leaves the rest unchanged.  Using this technique, we can freely move anywhere in the configuration space.

We have not directly shown that irreducible $\Gamma$-representations are unique, but since we know there are $2^{n^2}$ distinct irreducible $\Gamma$-representations, and we know that $\subgroup{n}$ is the entire space (meaning it also has size $2^{n^2}$), the irreducible $\Gamma$-representations \emph{must} be unique.  Not only that, but we actually have an algorithm for generating them.  For any matrix $M$, it is straightforward to express $M$ uniquely as a sum of distinct $\beta$ matrices.  We know the irreducible $\Gamma$-representation for each $\basis{i}{j}$: it is the sum on the left in \refx{sum_row_and_col}.  Substituting these representations for each $\beta$ in the $\beta$-representation of $M$, and we now have produced a $\Gamma$-representation of $M$.  It is then simply a matter of reducing.  Group like terms in that $\Gamma$-representation; terms that appear an even number of times cancel out, terms that appear on odd number of times may be reduced to a single appearance.  Thus, when $n$ is even, we have an algorithm for producing \emph{the} optimal winning game for any configuration of an $n$ by $n$ board. 

\section{Applications in Design and Play}
\label{applications}

\end{document}