\documentclass{article}[12pt]
\usepackage[margin=1in]{geometry}
\usepackage{amsmath}

\begin{document}

\section*{Introduction}

This document is to explain the mathematics behind the game CrossFade.  The rules of the game are simple: given a grid of lights, each of which is either on or off, attempt to turn all the lights off.  The mechanism is to "press" a given light, which will flip the on/off values of all the lights in the pressed light's row and column.

We will examine this game mathematically with the intent of providing understanding that will facilitate level design.  With this in mind, we state the following questions in advance:

\begin{itemize}
\item Are all initial configurations solvable?  If not, how can we determine which are solvable?
\item Does the order in which buttons are pressed matter when executing a solution?
\item Can we algorithmically determine solutions?
\end{itemize}

We will answer each of these questions in turn, using a little group theory.

\section*{Algebraic Framework}

To describe and manipulate the board and games, we will use matrices.  To keep our solution general, we will operate on an arbitrarily sized $n$x$n$ matrices, though small integers will be used when providing examples.  A game of CrossFade can be seen to take place inside the matrix group $M_{\mathbf{Z}/2\mathbf{Z}}(n,n)$.  That is, $n$ by $n$ matrices over the binary field $\mathbf{Z}/2\mathbf{Z}$.  This is fancy algebra speak for "$n$ by $n$ integer matrices mod 2."  Each configuration of the board can be seen as an $n$x$n$ matrix of 0's and 1's.  Such matrices can be added together (with the entries being added mod 2) to create further matrices of 0's and 1's.  Pressing the button at the $(i,j)^{\text{th}}$ position is equivalent to adding to the current board's matrix a matrix that is all zeros except the $i^{\text{th}}$ row and $j^{\text{th}}$ column, which are 1's.  For convenience, let's call these special matrices that represent moves $G_{i,j}$ (why we use $G$ will become clear in a moment):

$$
G_i,j = \left[ \begin{matrix}{cccccccc}
	0 & 0 & \cdots & 0 & 1 & 0 & \cdots & 0 \\
	0 & 0 & \cdots & 0 & 1 & 0 & \cdots & 0 \\
	\vdots & \vdots & \ddots & \vdots & vdots & \vdots & & \vdots \\
	0 & 0 & \cdots & 0 & 1 & 0 & \cdots & 0 \\
	1 & 1 & \cdots & 1 & 1 & 1 & \cdots & 1 \\
	0 & 0 & \cdots & 0 & 1 & 0 & \cdots & 0 \\
	\vdots & \vdots & & \vdots & \vdots & \vdots & & \vdots \\
	0 & 0 & \cdots & 0 & 1 & 0 & \cdots & 0 \\
	\end{matrix} \right]
$$

A game of CrossFade, then, is some initial board configuration $M_0$ and then a sequence of moves $(G_{i,j})_k$ that leads to the identity matrix, $I$ (the matrix of all 0's).  We can write a game out as a sum:
$$
M_0 + (G_{i,j})_1 + (G_{i,j})_2 + \cdots + (G_{i,j})_m =  I
$$
Where the $i,j$ in each move are understood to be different, though not necessarily unique, and $m$ is the length of the sequence of moves.  There are several insights we can gain from this structure.  Right off the bat, we can answer our second question: does order matter?  No.  Matrix addition is \emph{commutative}, meaning we can rearrange the terms at will and not change the result, just as we can with integer addition.  Therefore, the consequence of a set of moves depends only on the moves therein, not on the order in which they are made.  We can also come to a characterization of the matrices that represent winnable configurations by subtracting the $G_i,j$ to the other side.
$$
I - (G_{i,j})_1 - (G_{i,j})_2 - \cdots - (G_{i,j})_m = M_0
$$

Note now that in $M_{\mathbf{Z}/2\mathbf{Z}}(n,n), every matrix is its own negative: add a matrix to itself and the entries, added pairwise, will be either 0 or 2, so when we take the entire matrix mod 2, the result is the identity matrix.  Thus, every matrix is its own additive inverse, and we can simply replace the minus signs with plus signs:
$$
I + (G_{i,j})_1 + (G_{i,j})_2 + \cdots + (G_{i,j})_m = M_0
$$
We can draw several conclusions from this sum.  First, it verifies our intuition that a sequence of moves can be reversed by performing them all over again.